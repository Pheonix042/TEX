\documentclass{article}
\usepackage[utf8]{inputenc}
\usepackage{upgreek}
\usepackage{amsmath}
\usepackage{graphicx}
\usepackage{amssymb}
\usepackage{epstopdf}
\usepackage{color}
\usepackage{textcomp}
\usepackage{soul}
\newtheorem{theorem}{Theorem}
\usepackage{lmodern}
\usepackage{listings}
\usepackage{mathabx}
\usepackage{fixmath}
\usepackage{mathtools}
\def\undertilde#1{\mathord{\vtop{\ialign{##\crcr
$\hfil\displaystyle{#1}\hfil$\crcr\noalign{\kern1.5pt\nointerlineskip}
$\hfil\tilde{}\hfil$\crcr\noalign{\kern1.5pt}}}}}
\def\therefore{\boldsymbol{\text{ }
\leavevmode
\lower0.4ex\hbox{$\cdot$}
\kern-.4em\raise0.5ex\hbox{$\cdot$}
\kern-0.55em\lower0.4ex\hbox{$\cdot$}
\thinspace\text{ }}}
\hoffset=0pt\voffset=0pt\textwidth=520pt\textheight=740pt\topmargin=-70pt\oddsidemargin=-25pt\evensidemargin=20pt
\renewcommand\baselinestretch{1.2}
\newcommand{\inner}{\rule[-.01in]{.07in}{.005in}\vrule width.005in depth.01in height.07in\hspace{.03in}}

\title{Assignment-2 \\ ME-676A NL-FEM}
\author{Name:Romil Kadia, Roll No:16105045\\ Name:Ankur Maurya, Roll No:13124\\ Name:Rohit Kumawat, Roll No:13587}
\date{April 28,2017}

\begin{document}

\maketitle

\section*{Give Data}
Consider a elasto-plastic solid that obeys the von Misces yield condition
\begin{equation}
    \phi (\sigma , \sigma_{y}) = \sqrt{3 J_{2}} - \sigma_{y}\\
\end{equation}
\begin{equation}
    \sigma_{y} = \sigma_{0} + \bar{\epsilon_{p}} \bar{H}
\end{equation}
\section*{Mathematical Manipulation}
\begin{equation}
    A = \sigma_{y} = \frac{\partial \psi_{p}}{\partial \alpha}
\end{equation}
\begin{equation}
    \sigma=\frac{\partial \psi_{e}}{\partial \epsilon_{e}}
\end{equation}
We know that,
\begin{equation}
    H= - \frac{\partial \psi}{\partial \sigma_{y}}
\end{equation}
\begin{equation}
    N = \frac{\partial \psi}{\partial \sigma}
\end{equation}
Calculating some data prior so that that can be used later
\begin{equation}
    \frac{\partial^{2} \psi}{\partial \alpha^{2}}=\frac{\partial \sigma_{y}}{\partial \alpha} = \bar{H}
\end{equation}
We know that for Von-Misces:\\

$\alpha = \bar{\epsilon_{p}}$
\begin{equation}
    \frac{\partial \phi}{\partial A} = \frac{\partial \phi}{\partial \sigma_{y}}= -1
\end{equation}
\begin{equation}
    \Rightarrow H = - \frac{\partial \phi}{\partial A} = 1
\end{equation}
\pagebreak
\section*{For DDSDDE of continuum tangent stiffness matrix}
From the constitutive equation
\begin{equation}
    \dot{\sigma} = \undertilde{C}:(\dot{\epsilon}- \dot{\lambda} N)
\end{equation}
\begin{equation}
    \therefore \dot{\sigma}=(\undertilde{C}:\dot{\epsilon})-(\undertilde{C}:\dot{\lambda}N)
\end{equation}
For von-misces, using direct relation from notes:
\begin{equation}
    N= \sqrt{\frac{3}{2}} \frac{\sigma\prime}{||\sigma\prime||}
\end{equation}
We Know that when stress has reached yield surface then:\\
$\phi=0$
\begin{equation}
    \dot{\phi}=0=\frac{\partial \phi}{\partial \sigma}:\sigma + \frac{\phi}{\sigma_{y}}:\sigma_{y}
\end{equation}
\begin{equation}
    =\frac{\partial \phi}{\partial \sigma}: \undertilde{C}:(\dot{\epsilon} - \dot{\lambda} N) + \frac{\partial \phi}{\partial \sigma_{y}}:\dot{A}
\end{equation}
\begin{equation}
    =N:\undertilde{C}:\dot{\epsilon} - N:\undertilde{C}:\dot{\lambda}N + \frac{\partial \phi}{\partial \sigma_{y}}: \frac{\partial^{2} \phi}{\partial \sigma_{y}^{2}}:\dot{\alpha}
\end{equation}
\begin{equation}
    =N:\undertilde{C}:\dot{\epsilon} - N:\undertilde{C}:\dot{\lambda}N +\frac{\partial \phi}{\partial \sigma_{y}}:\frac{\partial^{2}}{\partial \sigma} :\dot{\lambda}H
\end{equation}
\begin{equation}
    \therefore 0 = N:\undertilde{C}:\dot{\epsilon} - (N:\undertilde{C}: N +\frac{\partial \phi}{\partial \sigma_{y}}:\frac{\partial^{2} \phi}{\partial \sigma_{y}^{2}}:H)\dot{\lambda} 
\end{equation}
\begin{equation}
    \Rightarrow \dot{\lambda} = \frac{N:\undertilde{C}:\dot{\epsilon}}{N: \undertilde{C}:N - \frac{\partial \phi}{\partial \sigma_{y}} :\frac{\partial^{2} \phi}{\partial \sigma_{y}^{2}}:H}
\end{equation}
Now, again for stress:
\begin{equation}
    \dot{\sigma}=\undertilde{C}:\dot{\epsilon}- \undertilde{C}:\dot{\lambda}N
\end{equation}
\begin{equation}
    = \undertilde{C}:\dot{\epsilon}- \frac{(\undertilde{C}:N)\otimes (\undertilde{C}:N)}{N:\undertilde{C}:N -\frac{\partial \phi}{\partial \sigma_{y}} :\frac{\partial^{2} \phi}{\partial \sigma_{y}^{2}}:H}
\end{equation}
And
\begin{equation}
    \undertilde{C}:N= \sqrt{\frac{3}{2}} \frac{2\mu \sigma\prime}{||\sigma\prime||^{2}}
\end{equation}
\begin{equation}
    \Rightarrow (\undertilde{C}:N)\otimes(\undertilde{C}:N)=\frac{6\mu^{2}}{||\sigma\prime||^{2}}(\sigma\prime\otimes\sigma\prime)
\end{equation}
But,
\begin{equation}
    \frac{\partial^{2} \phi}{\partial \epsilon_{p}^{2}} =\frac{\partial}{\partial \epsilon_{p}}(\frac{\partial \phi}{\partial \epsilon_{p}}) = \frac{\partial A}{\partial \epsilon_{p}}= \frac{\partial \sigma_{y}}{\partial \epsilon_{p}} = \bar{H}
\end{equation}
\begin{equation}
    \Rightarrow \undertilde{C_{ep}} =\undertilde{C}- \frac{6\mu^{2} (\sigma\prime\otimes\sigma\prime)}{2\mu(\sigma\prime:\sigma\prime) +\bar{H}(\sigma:\sigma)}
\end{equation}
But to completely define this we need to find cauchy stress and its deviatoric component\\
Given strain from abaqus
\begin{equation}
    \sigma_{ij}\prime= 2G\epsilon_{ij}\prime
\end{equation}
But,
\begin{equation}
    \epsilon_{ij}\prime= \epsilon_{ij} - \frac{\epsilon_{kk}\delta_{ij}}{3}
\end{equation}
\begin{equation}
    \Rightarrow \sigma_{ij}\prime = 2G(\epsilon_{ij} - \frac{\epsilon_{kk}\delta_{ij}}{3})
\end{equation}
\begin{equation}
    \sigma_{ij}\prime\otimes\sigma_{ij}\prime= 4G^{2}(\epsilon_{ij} - \frac{\epsilon_{kk}\delta_{ij}}{3})\otimes(\epsilon_{ij} - \frac{\epsilon_{kk}\delta_{ij}}{3})
\end{equation}
Similarly for Cauchy stress
\begin{equation}
    \sigma_{ij}=\sigma_{ij}\prime + \sigma_{kk}\frac{\delta_{ij}}{3}
\end{equation}
\begin{equation}
    \Rightarrow C_{ijkl}^{ep}= C_{ijkl} - \frac{6\mu^{2} \sigma_{ij}\prime\sigma_{kl}\prime}{\bar{H}(\sigma_{cd}\sigma_{cd}) +3\mu(\sigma_{ab}\prime\sigma_{ab}\prime)}
\end{equation}
let, K=Constant
\begin{equation}
    K=\frac{6\mu^{2}}{\bar{H}(\sigma_{cd}\sigma_{cd}) +3\mu(\sigma_{ab} \prime\sigma_{ab}\prime)}
\end{equation}
\section*{Answer-1(A)}
DDSDDE in voight notation is as follows:\\
DDSDDE=
\begin{bmatrix}
$\lambda +2\mu -K\sigma_{1}\prime\sigma_{1}\prime & \lambda-K\sigma_{1}\prime\sigma_{2}\prime & \lambda- K\sigma_{1}\prime\sigma_{3}\prime & -K\sigma_{1}\prime\sigma_{4}\prime & -K\sigma_{1}\prime\sigma_{5}\prime & -K\sigma_{1}\prime\sigma_{6}\prime$\\
$--& \lambda +2\mu -K\sigma_{2}\prime\sigma_{2}\prime & \lambda-K\sigma_{3}\prime\sigma_{2}\prime & -K\sigma_{2}\prime\sigma_{4}\prime  & -K\sigma_{2}\prime\sigma_{5}\prime & -K\sigma_{2}\prime\sigma_{6}\prime$\\
$--&--& \lambda +2\mu -K\sigma_{3}\prime\sigma_{3}\prime & -K\sigma_{3}\prime\sigma_{4}\prime & -K\sigma_{3}\prime\sigma_{5}\prime & -K\sigma_{3}\prime\sigma_{6}\prime$\\
$--&--&--& \mu-K\sigma_{4}\prime\sigma_{4}\prime & -K\sigma_{4}\prime\sigma_{5}\prime & -K\sigma_{4}\prime\sigma_{6}\prime$\\
$--&--&--&--& \mu-K\sigma_{5}\prime\sigma_{5}\prime & -K\sigma_{5}\prime\sigma_{6}\prime$\\
$sym&--&--&--&--& \mu-K\sigma_{6}\prime\sigma_{6}\prime$\\
\end{bmatrix}
\pagebreak
\section*{Consistent tangent stiffness}
From Abaqus we will get the value of strain and Dstrain and state variavle which will give plastic strain, all this values will be provided for time($t_{n}$)\\
Deriving some useful relations:\\
from equation-18
\begin{equation}
    \dot{\bar{\epsilon_{p}}}= -\dot{\lambda}
\end{equation}
\begin{equation}
    \epsilon^{t}_{n+1}= \epsilon^{t}_{n} + \bigtriangleup\epsilon
\end{equation}
From now onwards all the values are calculated for time (n+1)\\
so, no need to mention (n+1) as subscript everywhere
\begin{equation}
    \sigma^{t}_{n+1}=\undertilde{C}:\epsilon^{t}_{n+1}
\end{equation}
\begin{equation}
    \sigma^{t}\prime=2G \epsilon^{t}\prime
\end{equation}
\begin{equation}
    J_{2}=0.5(\sigma^{t}\prime:\sigma^{t}\prime)
\end{equation}
And,
\begin{equation}
    \phi = \sqrt{\frac{3}{2}(\sigma^{t}\prime:\sigma^{t}\prime)}-\sigma_{y}
\end{equation}
Proceeding with radial return algorithm and assuming that $\phi \geq 0$\\
Next three equation are coupled which need to be solved to proceed further in radial return algo\\
\begin{equation}
    \epsilon_{e}=\epsilon^{t}_{e} -\bigtriangleup\lambda \sqrt{\frac{3}{2}}\frac{\sigma^{t}\prime}{||\sigma^{t}\prime||}
\end{equation}
\begin{equation}
    \epsilon_{p}= \epsilon_{p(n)}+\bigtriangleup\lambda
\end{equation}
\begin{equation}
    \phi_{(n+1)}=0
\end{equation}
But, from equation no. 35, equation 38 and 39 will be modified as
\begin{equation}
     \epsilon_{e}=\epsilon^{t}_{e} -\bigtriangleup\lambda \sqrt{\frac{3}{2}}\frac{1}{2G}\frac{\epsilon^{t}\prime}{||\epsilon^{t}\prime||}
\end{equation}
\begin{equation}
    \phi = \sqrt{\frac{3}{2}(\epsilon^{t}\prime:\epsilon^{t}\prime)}-\sigma_{y}
\end{equation}
Breaking above eq.38 in volumetric and deviatoric components leads to:
\begin{equation}
    \epsilon_{e}\prime=\epsilon^{t}_{e}\prime -\bigtriangleup\lambda \sqrt{\frac{3}{2}}\frac{1}{2G}\frac{\epsilon^{t}\prime}{||\epsilon^{t}\prime||}
\end{equation}
\begin{equation}
    \epsilon_{v}=\epsilon^{t}_{v}
\end{equation}
Further using this eq.43 and 44 to find respective components of stress, leads to:\\
By using linear relation
\begin{equation}
    \frac{\sigma\prime}{||\sigma\prime||}=\frac{\sigma^{t}\prime}{||\sigma^{t}\prime||}
\end{equation}
\begin{equation}
  \sigma\prime=(1-\sqrt{\frac{3}{2}} \frac{2\bigtriangleup\lambda G}{||\sigma^{t}\prime||})  \sigma^{t}\prime
\end{equation}let,
\begin{equation}
    q^{t}=\sqrt{3J_{2}}
\end{equation}
\begin{equation}
    =\sqrt{3 \sigma^{t}\prime:\sigma^{t}\prime}
\end{equation}
\begin{equation}
    \Rightarrow \sigma\prime=(1-\sqrt{\frac{3}{2}} \frac{3\bigtriangleup\lambda G}{q^{t}})  \sigma^{t}\prime
\end{equation}

\pagebreak
Final stress update equation is given by:
\begin{equation}
    \sigma_{(n+1)}=\sigma\prime_{(n+1)} + P_{(n+1)}
\end{equation}
\begin{equation}
    = (1-\sqrt{\frac{3}{2}} \frac{3\bigtriangleup\lambda G}{q^{t}})  \sigma^{t}\prime + k\epsilon_{v(n+1)}
\end{equation}
\begin{equation}
    =(1-\sqrt{\frac{3}{2}} \frac{3\bigtriangleup\lambda G}{q^{t}})2G \epsilon^{t}\prime +k\epsilon_{v(n+1)}
\end{equation}
By using the relation for fourth order identity tensor to obtain final expression:
\begin{equation}
    A\prime = I_{d}: A
\end{equation}
where,
\begin{equation}
    I_{d}=\frac{1}{2}(\delta_{ik}\delta_{jl}+\delta_{il}\delta_{jk})-(\frac{1}{3}\delta_{ij}\delta_{kl})
\end{equation}
\begin{equation}
    \Rightarrow \sigma_{(n+1)}=\undertilde{C}:\epsilon^{t} + (\undertilde{H(\phi)}I_{d}\frac{6G^{2} \bigtriangleup\lambda}{q_{t}}):\epsilon^{t}
\end{equation}
where, $\undertilde{H(\phi)}$ is heaviside function\\
we are neglecting heaviside function as this relation is used to derive tangent stiffness matrix knowing that $\phi \geq 0$\\
using inditial notation now
\begin{equation}
    \therefore \sigma_{ij}= C_{ijkl}\epsilon^{t}_{kl} -\frac{6G^{2} \bigtriangleup\lambda}{q^{t}}(\frac{\delta_{ik}\delta_{jl} \epsilon^{t}_{kl}}{2} +\frac{\delta_{il}\delta_{jk} \epsilon^{t}_{kl}}{2}- \frac{\delta_{ij}\delta_{kl} \epsilon^{t}_{kl}}{3})
\end{equation}
\begin{equation}
    = C_{ijkl}\epsilon^{t}_{kl} -3G^{2}\frac{\bigtriangleup\lambda}{q^{t}}(\epsilon^{t}_{ij}+\epsilon^{t}_{ji}) +2G^{2}\frac{\bigtriangleup\lambda}{q^{t}}\delta_{ij}\delta_{kl}\epsilon^{t}_{kl}
\end{equation}
\begin{equation}
    = C_{ijkl}\epsilon^{t}_{kl}-6G^{2}\frac{\bigtriangleup\lambda}{q^{t}}\epsilon^{t}_{ij}+2G^{2}\frac{\bigtriangleup\lambda}{q^{t}}\delta_{ij}\epsilon^{t}_{kl}
\end{equation}
Deriving some useful relations which are going to be required in further part of derivation
\begin{equation}
    q^{t}=2G\sqrt{3\epsilon^{t}\prime:\epsilon^{t}\prime}
\end{equation}
\begin{equation}
    =2G\sqrt{3(\epsilon_{ij}-\frac{\epsilon_{mm}\delta_{ij}}{3})(\epsilon_{ij}-\frac{\epsilon_{kk}\delta_{ij}}{3})}
\end{equation}
\begin{equation}
    =2G\sqrt{3(\epsilon^{t}_{ij}\epsilon^{t}_{ij}- \frac{2}{3}\epsilon^{2}_{kk} +\frac{1}{3} \epsilon^{2}_{kk})}
\end{equation}
\begin{equation}
    \therefore q^{t}=2G\sqrt{3(\epsilon^{t}_{ij}\epsilon^{t}_{ij}- \frac{1}{3}\epsilon^{2}_{kk})}
\end{equation}
using the eq. 45 and 49
\begin{equation}
    q^{t} = (3G+\bar{H})\bigtriangleup\lambda + \bar{H}\bar{\epsilon_{(n)}}+\sigma_{0}
\end{equation}
\begin{equation}
    \therefore \bigtriangleup\lambda = \frac{q^{t}+ \bar{H}\bar{\epsilon_{(n)}}}{3G+\bar{H}}
\end{equation}
Now we will differentiate eq. 63 and 64 w.r.t $\epsilon^{t}_{kl}$ which is elastic only
\begin{equation}
    \frac{\partial q^{t}}{\partial \epsilon^{t}_{kl}}=\frac{G\sqrt{3}}{\sqrt{3(\epsilon^{t}_{ij}\epsilon^{t}_{ij}- \frac{1}{3}\epsilon^{2}_{kk})}}(\epsilon^{t}_{ij}(\delta_{ik}\delta_{jl} + \delta_{il}\delta_{jk}) - \frac{2\epsilon^{t}_{mm}}{3} (\delta_{ij}\delta_{kl}))
\end{equation}
and in the numerator portion there are two deviatoric strain trial as well as in denominator its double contraction of deviatoric strain\\
hence,
\begin{equation}
    \frac{\partial q^{t}}{\partial \epsilon^{t}_{kl}}= \frac{2\sqrt{3}G \epsilon^{t}_{kl}}{\sqrt{\epsilon^{t}_{mn}\epsilon^{t}_{mn}}}
\end{equation}

\pagebreak
\begin{equation}
    \frac{\partial \bigtriangleup\lambda}{\epsilon^{t}_{kl}}= \frac{1}{3G+\bar{H}}(\frac{\sqrt{3}G\epsilon^{t}_{kl}\prime}{\sqrt{\epsilon^{t}_{mn}\epsilon^{t}_{mn}}} -\bar{H}\frac{\partial \bar{\epsilon_{p}}}{\epsilon^{t}_{kl}})
\end{equation}
here first term in bracket is $\frac{\partial q^{t}}{\partial \epsilon^{t}_{kl}}$
\begin{equation}
    \therefore  \frac{\partial \bigtriangleup\lambda}{\epsilon^{t}_{kl}}= \frac{1}{3G+\bar{H}} \frac{\sqrt{3}G\epsilon_{kl}\prime}{\sqrt{\epsilon^{t}_{mn}\epsilon^{t}_{mn}}}
\end{equation}
\section*{Differentiation of eq. 58 w.r.t $\epsilon_{e(kl)}$}
\begin{equation}
    \frac{\partial \sigma_{ij}}{\partial \epsilon^{t}_{kl}}= (1) +(2)+(3)
\end{equation}
calculating each component separately:
\begin{equation}
    (1)= C_{ijkl}
\end{equation}
\begin{equation}
    (2)= 6G^{2}(\frac{\bigtriangleup\lambda}{q^{t}}\frac{\delta_{ik}\delta_{jl}+\delta_{il}\delta_{jk}}{2}) +\epsilon^{t}_{ij}(\frac{1}{q^{t}(3G+\bar{H})}  \frac{\partial q^{t}}{\partial \epsilon^{t}_{kl}})- \frac{\bigtriangleup\lambda}{(q^{t})^{2}}\frac{\partial q^{t}}{\epsilon^{t}_{kl}}
\end{equation}
\begin{equation}
    =\frac{6G^{2}}{q^{t}}(\frac{\bigtriangleup\lambda}{2}(\delta_{ik}\delta_{jl}+\delta_{il}\delta_{jk}) + \frac{\epsilon^{t}_{ij}}{2(3G+\bar{H})}\frac{\partial q^{t}}{\epsilon^{t}_{kl}} +\frac{\bigtriangleup\lambda \epsilon^{t}_{ij}}{q^{t}}\frac{\partial q^{t}}{\epsilon^{t}_{kl}})
\end{equation}
\begin{equation}
    (3)=2G^{2}\delta_{ij}[\frac{\epsilon^{t}_{kk}}{q^{t}} 
    \frac{q^{t}}{2(3G+\bar{H})}\frac{\partial q^{t}}{\epsilon^{t}_{kl}}
    \bigtriangleup\lambda \frac{\partial q^{t}}{\epsilon^{t}_{kl}}] +\frac{\bigtriangleup\lambda}{q^{t}}(2\delta_{zz}\delta_{kl})
\end{equation}
\begin{equation}
    C^{ep}_{ijkl}=C_{ijkl}-3G^{2}\frac{\bigtriangleup\lambda}{q^{t}}(\delta_{ik}\delta_{jl}+\delta_{il}\delta_{jk})
    +(12G^{2}\delta_{ij}\delta_{kl}\frac{\bigtriangleup\lambda}{q^{t}}
    +\frac{6g^{2}\bigtriangleup\lambda}{(q^{t})^{2}}\epsilon^{t}_{ij}
    +\frac{G^{2}\delta_{ij}\epsilon_{zz}}{q^{t}(3G+\bar{H})}
    -\frac{3G^{2}}{3G+\bar{H}}\frac{\epsilon^{t}_{ij}}{q^{t}}
    2G^{2}\epsilon_{zz}\delta_{ij}\frac{\bigtriangleup\lambda}{(q^{t})^{2}})\frac{\partial q^{t}}{\epsilon^{t}_{kl}}
\end{equation}
Now to write DDSDDE in matrix form using voight notation\\
we will define some scalars:
\begin{equation}
    QQ=q^{t}
\end{equation}
\begin{equation}
    lam=\bigtriangleup\lambda
\end{equation}
\begin{equation}
    \frac{\partial q^{t}}{\epsilon^{t}_{kl}}=\epsilon^{t}\prime (QCON)
\end{equation}
\begin{equation}
    \frac{\partial \bigtriangleup\lambda}{\epsilon^{t}_{kl}}=\epsilon^{t}\prime(LAMCON)
\end{equation}
\begin{equation}
    \epsilon_{zz}=VOL
\end{equation}
\begin{equation}
    \frac{G^{2}}{3G+\bar{H}}=GCON
\end{equation}
\begin{equation}
    B_{1}=-\frac{3G^{2} LAM}{QQ}
\end{equation}
\begin{equation}
    B_{2}=\frac{3(GCON)(QCON)}{QQ}
\end{equation}
\begin{equation}
    B_{3}=\frac{6G^{2}(LAM)(QCON)}{QQ^{2}}
\end{equation}
\begin{equation}
    B_{4}=\frac{12G^{2}(LAM)}{QQ}
\end{equation}
\begin{equation}
    B_{5}=\frac{(VOL)(GCON)(QCON)}{QQ}
\end{equation}
\begin{equation}
    B_{6}=\frac{2G^{2}(LAM)(QCON)(VOL)}{QQ^{2}}
\end{equation}
\begin{equation}
    X=-B_{2}+B_{3}
\end{equation}
\begin{equation}
    Y=B_{5}+B_{6}
\end{equation}
\begin{equation}
    Z=\lambda+2\mu +2B_{1}+B_{4}
\end{equation}
\begin{equation}
    W=\mu + B_{1}
\end{equation}
Using eq. from 75 to 88 in eq. 74, leads to simplified form of eq. 74:
\begin{equation}
    C^{ep}_{ijkl}=C_{ijkl}+B_{1}(\delta_{ik}\delta_{jl}+\delta_{il}\delta_{jk}) +X\epsilon^{t}_{ij}\epsilon^{t}_{kl}\prime +Y\delta_{ij}\epsilon^{t}_{kl}\prime +B_{4}\delta_{ij}\delta_{kl}
\end{equation}
\section*{Answer-1(B)}
DDSDDE in matrix is fiven by:\\
\begin{bmatrix}
$(X\epsilon_{1}+Y)\epsilon_{1}\prime & 
\lambda+B_{4}+(X\epsilon_{1}+Y)\epsilon_{2}\prime & \lambda+B_{4}+(X\epsilon_{1}+Y)\epsilon_{3}\prime & (X\epsilon_{1}+Y)\epsilon_{4}\prime & 
(X\epsilon_{1}+Y)\epsilon_{5}\prime & 
(X\epsilon_{1}+Y)\epsilon_{6}\prime$\\
$-- & Z+(X\epsilon_{2}+Y)\epsilon_{2}\prime & 
\lambda+B_{4}+(X\epsilon_{2}+Y)\epsilon_{3}\prime & 
(X\epsilon_{2}+Y)\epsilon_{4}\prime & 
(X\epsilon_{2}+Y)\epsilon_{5}\prime & 
(X\epsilon_{2}+Y)\epsilon_{6}\prime$\\
$-- & -- & Z+(X\epsilon_{3}+Y)\epsilon_{3}\prime & 
(X\epsilon_{3}+Y)\epsilon_{4}\prime & 
(X\epsilon_{3}+Y)\epsilon_{5}\prime & 
(X\epsilon_{3}+Y)\epsilon_{6}\prime$\\
$-- & -- & -- & W+X\epsilon_{4}\epsilon_{4}\prime & X\epsilon_{4}\epsilon_{5}\prime & X\epsilon_{4}\epsilon_{6}\prime$\\
$-- & -- & -- & -- & W+X\epsilon_{5}\epsilon_{5}\prime & X\epsilon_{5}\epsilon_{6}\prime$\\
$-- & -- & -- & -- & -- & W+X\epsilon_{6}\epsilon_{6}\prime$\\
\end{bmatrix}
\section*{Answer-1(C)}
Expression for $\bigtriangleup\lambda$ from eq. 64
\begin{equation}
    \bigtriangleup\lambda = \frac{q^{t}+ \bar{H}\bar{\epsilon_{(n)}}}{3G+\bar{H}}
\end{equation}
\end{document}
