\documentclass{article}
\usepackage[utf8]{inputenc}
\usepackage{upgreek}
\usepackage{amsmath}
\usepackage{listings}
\usepackage{fullpage}
\usepackage[normalem]{ulem}
\def\undertilde#1{\mathord{\vtop{\ialign{##\crcr
$\hfil\displaystyle{#1}\hfil$\crcr\noalign{\kern1.5pt\nointerlineskip}
$\hfil\tilde{}\hfil$\crcr\noalign{\kern1.5pt}}}}}
\def\therefore{\boldsymbol{\text{ }
\leavevmode
\lower0.4ex\hbox{$\cdot$}
\kern-.4em\raise0.5ex\hbox{$\cdot$}
\kern-0.55em\lower0.4ex\hbox{$\cdot$}
\thinspace\text{ }}}
\title{Assignment-1 ME-676A}
\author{Name:Romil Kadia, Roll No:16105045 && Name: Ankur Maurya, Roll No: 13124 && Name: Rohit Kumawat Roll No: 13587}
\date{April 12, 2017}
\usepackage{natbib}
\begin{document}
\maketitle
\section{Problem-1(A)}
DDSDDE for abaqus is the tangent stiffness matrix for Jauman rate of kirchoff stress $\undertilde{\uptau}^\nabla$\\
DDSDDE in voight notation used by ABAQUS is as follows:\\
\\
DDSDDE$=\frac{1}{J} \frac{\partial {\uptau}^\nabla}{\partial D \epsilon}$\\
\\
DDSDDE=
\begin{bmatrix}
$\lambda J^{-1}+ \frac{2 \mu B_{11}}{J} & \frac{\lambda}{J} & \frac{\lambda}{J} & \frac{\mu B_{12}}{J} & \frac{\mu B_{13}}{J} & 0 $\\
\\
$ -- & \frac{\lambda +2\mu B_{22}}{J} & \frac{\lambda}{J} & \frac{\mu}{J} B_{21} & 0 & \frac{\mu}{J} B_{23}$\\
\\
$ -- & -- & \frac{\lambda + 2\mu B_{33}}{J} & 0 & \frac{\mu}{J} B_{31} & \frac{\mu}{J} B_{32}$\\
\\
$ -- & -- & -- & \frac{\mu}{2J} (B_{11} + B_{22}) & \frac{\mu}{2J} B_{23} & \frac{\mu}{2J} B_{13}$\\
\\
$ -- & -- & -- & -- & \frac{\mu}{2J} (B_{11} + B_{33}) & \frac{\mu}{2J} B_{12}$\\
\\
$ sym & -- & -- & -- & -- & \frac{\mu}{2J} (B_{22} + B_{33})$
\end{bmatrix}

\section{Problem-1(B)}
\uline{To update stress at every step is done as follows:}\\

1) At the end of each step udation in deformation tensor is done by abaqus\\

2) Then we are going to update BB matrix which is $F^T F$\\

$\undertilde{B} = \undertilde{F^T} \undertilde{F}$\\

3) Hence stress will be updated as follows:\\
\\

$\sigma_{ij}=\frac{\mu}{J} B_{ij} + \frac{(\lambda ln J - \mu)}{J}\delta_{ij}$\\
\\

\uline{above equation in voight notation:}\\

for i=1,2,3\\

$\sigma_{i}=\frac{\mu}{J} B_{i} + \frac{(\lambda ln J - \mu)}{J}$\\

and for i=4,5,6\\

$\sigma_{i}=\frac{\mu}{J} B_{i}$
\section{Problem-1(C)}
\uline{Complete UMAT code for user defined material hyperelastic Neo-Hookean materrial in as follows with two elastic constant $\mu$ and $\lambda$:}\\
\lstinputlisting[language=Fortran]{work2.f}
\clearpage
\section{Problem-2}
Consider Neo-Hookean material as incompressible then,\\

J=1\\

\therefore $\lambda_{1} \lambda_{2} \lambda_{3} = 1$\\
\\
and for the uniaxial case\\

$\lambda_{2}= \lambda_{3}$\\

and, $\lambda_{2}= \lambda_{3} = \frac{1}{\sqrt{\lambda_{1}}}$\\
\\
Calculating $\undertilde{B}$\\
assuming that deformation gradient is in eigen basis\\

\therefore $[F]=
\begin{bmatrix}
$\lambda_{1} & 0 & 0$\\
$0 & \lambda_{2} & 0$\\
$0 & 0 & \lambda_{3}$\\
\end{bmatrix}$\\

and $\undertilde{B}=\undertilde{F}\undertilde{F^T}$\\

\therefore [B]=
$
\begin{bmatrix}
$\lambda^2_{1} & 0 & 0$\\
$0 & \lambda^2_{2} & 0$\\
$0 & 0 & \lambda^2_{3}$\\
\end{bmatrix}
$\\
\\
\uline{Modefiying expression of energy density:}\\

for $J=1$\\

$\psi_{0} = \frac{\mu}{2} (\lambda^2_{1} + \lambda^2_{2} + \lambda^2_{3}-3)$\\
\\
because, $I_{1}$ for $\undertilde{C}$ is same as $I_{1}$ for $\undertilde{B}$ as both are diagonal matrix and $ln J=0$ \\

But, $\lambda_{2}=\lambda_{3}$\\
\\

\therefore $\psi_{0}=\frac{\mu}{2}(\lambda^2_{1}+2 \lambda^2_{2}-3)$\\


$ = \frac{\mu}{2}(\lambda^2_{1}+\frac{2}{\lambda_{1}}-3)$\\
\\
Now, for incompressible neo-hookean material Principal cauchy stress is given by:\\
\\

$\sigma_{i} = \frac{\lambda_{i}}{\lambda_{1} \lambda_{2} \lambda_{3}} \frac{\partial \psi_{0}}{\partial \lambda_{i}} + p $\\

where, p is undetermined pressure\\
\\

And, $\frac{\partial \psi_{0}}{\partial \lambda_{1}} = \frac{\mu}{2}(\lambda^2_{1}- \frac{2}{\lambda^2_{1}}) $\\


$\frac{\partial \psi_{0}}{\partial \lambda_{2}} =\frac{\partial \psi_{0}}{\partial \lambda_{3}} = 0$\\
\\
Now, in voight notation\\

$\sigma_{1} = \lambda_{1}\frac{\partial \psi_{0}}{\partial \lambda_{1}} + p$\\

\therefore$\sigma_{1}= \mu (\lambda^2_{1} - \frac{1}{\lambda_{1}})+p$\\
\\

And $\sigma_{2} = \sigma_{3}= p$\\
\\
Now, $\sigma_{1}-\sigma_{2} = \lambda_{1}\frac{\partial \psi_{0}}{\partial \lambda_{1}}$\\
\\
But, as per given data $\sigma_{2} = \sigma_{3}=0$\\
\\
Hence, \sigma_{1} = \lambda_{1}\frac{\partial \psi_{0}}{\partial \lambda_{1}}
\\
\\
$Final expression for Principal Stress in $e_{1}$ direction i.e. $\sigma_{11}$$\\


$\sigma_{1}= \mu (\lambda^2_{1} - \frac{1}{\lambda_{1}})$
\section{Problem-3(A)}
*.inp file for abaqus to run above UMAT code is as follows:\\
\\
\uline{*inp file for case $\mu = 1.50 $ and $\lambda = 1100$, as per fiven data}\\
\lstinputlisting[language=Fortran]{j1.inp}
\\
\\
\uline{To plot graph comparing the abaqus results and analytical results(for incompressible case) are obtained as follows:}\\

1) Take data for principal stress $\sigma_{11}$ and principal strain $E_{11}$\\

2) using the following expression for calculating primary stretch $\lambda_{1}$\\

$\lambda_{1}=\sqrt{1+2*E_{11}}$\\

3) analytical expression for stress stretch is as follows:\\

$\sigma_{1}= \mu (\lambda^2_{1} - \frac{1}{\lambda_{1}})$\\

4) from step-1,2 and step-3 we will get abaqus result plot and analytical result(for incompressible case)

plot respectively using MatLab:\\

\lstinputlisting[language=Matlab]{abaplot.m}

\end{document}
